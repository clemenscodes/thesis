\begin{abstract}
	This thesis explores the comparative use of functional package management and
	containerization technologies, focusing on their applicability to development
	environments and software deployment. The motivation behind this research is the
	increasing need for reproducible, efficient, and scalable software environments in
	modern development workflows. As organizations scale and applications become more
	complex, choosing the right tools for managing development and deployment processes
	becomes critical.

	The core research question investigated is how Nix, a functional package manager,
	and Docker, a containerization platform, compare in terms of efficiency, scalability,
	and reproducibility in different software development and deployment contexts. The
	thesis seeks to determine which tool is more suited for specific use cases, such as
	small-scale, self-hosted environments versus large-scale, high-traffic applications.
	Through an empirical evaluation of development workflows and deployment models,
	the research highlights key differences in how each tool handles package isolation,
	process isolation, scalability, and infrastructure management.

	Methodologically, the research was conducted through hands-on experimentation with
	both Docker and Nix, comparing their performance in setting up reproducible
	development environments and deploying software. The study also analyzed the use
	of binary caches and declarative environments in Nix, as well as Docker's
	container-based isolation and scalability with orchestration tools like Kubernetes.

	The findings suggest that Nix is highly effective for setting up reproducible
	development environments and is particularly beneficial for small-scale deployments
	where traffic is predictable and resources are limited. Docker, on the other hand,
	excels in large-scale deployments and high-traffic applications due to its
	scalability and flexibility in dynamic environments. Despite Docker's advantages
	in scaling, Nix's use for development environments remains valuable even in larger
	projects due to its time-saving benefits and reduced friction in managing dependencies.

	The implications of this research extend to both practice and academic study.
	Practitioners can leverage the findings to better choose tools based on the specific
	needs of their projects, whether prioritizing reproducibility or scalability. From
	an academic perspective, this thesis contributes to the growing body of research
	on software deployment and environment management, suggesting areas where further
	integration between functional package management and containerization technologies
	could be explored.
\end{abstract}

\cleardoublepage
