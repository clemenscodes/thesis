\begin{abstract}
	This thesis explores the comparative use of functional package management and
	containerization technologies, focusing on their applicability in development
	environments and software deployment. As software projects scale and grow more
	complex, the need for reproducible, efficient, and scalable environments has
	become critical for modern workflows. Selecting the appropriate tools for
	managing these environments can significantly impact development productivity
	and deployment performance.

	The research compares functional package management and containerization
	in terms of efficiency, scalability, and reproducibility across different
	contexts, ranging from small-scale, self-hosted environments to large-scale,
	high-traffic applications. It highlights how these technologies manage package
	isolation, process isolation, and dynamic scaling, illustrating their strengths
	and trade-offs in development and deployment workflows.

	The findings demonstrate that functional package management is highly effective
	in creating reproducible and efficient development environments. By enabling
	precise control over dependencies, it minimizes configuration drift and friction
	during development, making it particularly beneficial for self-hosted servers, 
  personal desktops and stable traffic environments. 
  This approach also excels in deployments where
	resource efficiency is prioritized over dynamic scaling, offering a simpler
	solution without requiring complex orchestration.

	Conversely, containerization proves advantageous in large-scale, distributed
	applications where dynamic scalability is essential. The technology's ability
	to isolate services and leverage orchestration tools makes it ideal for handling
	fluctuating traffic demands, microservice architectures, and cloud-native
	environments. However, in development environments, containerization can
	introduce unnecessary overhead and complexity, where functional package management
	would be a more efficient choice.

	An important recommendation of this thesis is to prioritize functional package
	management for development environments due to its streamlined setup and resource
	efficiency. Startups and organizations should evaluate their deployment needs,
	opting for containerized solutions only when scalability and service orchestration
	are critical, while leveraging functional package management for simpler and
	more predictable workloads.

	The implications of this research are relevant for both practitioners and academic
	inquiry, offering insights into selecting the appropriate technologies based on
	specific project needs. It also opens up avenues for exploring hybrid solutions
	that combine the reproducibility of functional package management with the
	scalability of containerization, presenting an opportunity for future integration
	and innovation in software deployment strategies.
\end{abstract}

\cleardoublepage
