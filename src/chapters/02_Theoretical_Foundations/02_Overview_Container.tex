\section{Containerization Technologies}

This section introduces the fundamental concepts, principles, and architectural
components of containerization. Containerization technologies have revolutionized
the way software is developed, shipped, and deployed. By encapsulating applications
and their dependencies into isolated containers, these technologies ensure that
software runs consistently across various environments \cite[Under the Hood]
{merkelDockerLightweightLinux2014}.

\subsection{Definition and Scope}

Containerization is a lightweight form of virtualization \cite[Containers vs. Other
	Types of Virtualization]{merkelDockerLightweightLinux2014} that packages an
application along with its dependencies, libraries, and configuration files into a
single container. Unlike traditional virtual machines, containers share the host
operating system's kernel but run as isolated processes \cite[Containers vs. Other
	Types of Virtualization]{merkelDockerLightweightLinux2014}. This isolation provides
a consistent runtime environment, ensuring that applications behave the same way
regardless of where they are deployed.
The scope of containerization encompasses several key aspects. First, isolation is
a fundamental feature where containers provide process and resource isolation
through mechanisms like namespaces \cite[Under the Hood]{merkelDockerLightweightLinux2014}
and control groups (cgroups) \cite[Under the Hood]{merkelDockerLightweightLinux2014}
in the Linux kernel. This ensures that containers do not interfere with each other,
even though they share the same operating system kernel \cite{OperatingSystem2024}.

Portability is another crucial aspect \cite{SoftwarePortability2024}. Containers can
run on any system that supports the container runtime, making them highly portable
across different environments, such as development, testing, and production.
Additionally, containerization offers significant efficiency advantages. Containers
are more resource-efficient compared to traditional virtual machines because they
do not require a full OS instance for each application \cite[Under the Hood]
{merkelDockerLightweightLinux2014}. Instead, they share the host system's kernel,
which leads to lower overhead and faster start-up times.

Containerization has become an essential technology in modern software development
\cite[Docker: a Little Background]{merkelDockerLightweightLinux2014}, enabling the
development and deployment of applications in a consistent, repeatable manner.

\subsection{Containerization Systems}

Several containerization systems exemplify the principles and benefits of this
technology, providing robust tools for developing, shipping, and running applications
in a consistent and isolated manner.

Docker \cite{DockerHomepage2022} is one of the most widely used container platforms.
It offers a comprehensive suite of tools for building, sharing, and running containers.
Docker simplifies the process of creating container images with Dockerfiles and
provides a powerful runtime for managing containers \cite{DockerfileReference0200}.
Its popularity has led to a vast ecosystem of pre-built images available on Docker
Hub, making it easy to find and use a wide variety of software \cite{DockerHubContainer}.

Another notable example is Podman \cite{Podman2024}, a daemonless container engine
developed by Red Hat \cite{RedHatWe}. Podman provides a similar user experience to
Docker but does not require a background daemon, enhancing security and flexibility.
It supports rootless containers, allowing users to run containers without elevated
privileges, which improves security by reducing the attack surface \cite[4.1]
{priedhorskyMinimizingPrivilegeBuilding2021}.

These examples illustrate the diverse range of containerization systems available,
each with its unique features and strengths. Whether it's Docker's ease of use or
Podman's security features, these systems demonstrate the versatility and effectiveness
of containerization in modern software development and deployment.

\subsection{Theoretical Concepts}

Containerization technologies, such as Docker and Podman, have revolutionized
the software deployment landscape by offering lightweight, efficient methods
for isolating applications and their dependencies. The core theoretical
concepts underlying containerization are rooted in OS-level isolation and
resource control mechanisms that provide environments capable of running
applications consistently across various infrastructure environments.
A central concept in containerization is process isolation, which is
achieved through the use of namespaces. Namespaces ensure that
applications running within a container are isolated from processes on the
host or in other containers. By providing isolated views of the system,
namespaces prevent containers from interfering with each other or accessing
resources not explicitly shared \cite{Cgroups2024}. This isolation allows
containers to simulate separate operating systems while still sharing the
host's kernel.
Another critical concept is resource control, which is primarily
implemented through control groups (cgroups). Cgroups allow
the system to limit, prioritize, and isolate resource usage (e.g., CPU, memory,
and network bandwidth) for processes within a container. This ensures that
containers do not exceed their allocated resources, thus maintaining system
stability even when multiple containers run concurrently
\cite[Under the Hood]{merkelDockerLightweightLinux2014}.
Containers also rely on image layering, a technique that allows
reusable and immutable layers of a container image to be shared across
different containers. This reduces redundancy and improves efficiency, as
containers only need to store and transfer the unique layers for each
application \cite[Chapter 2.4]{merkelDockerLightweightLinux2014}. Layers
improve the overall speed of container deployment and resource usage,
enabling faster initialization and scaling in dynamic environments.
While containers offer significant benefits in terms of isolation and resource
control, the concept of reproducibility remains a challenge. Unlike
functional package management systems, containers rely on external base images
(e.g., Ubuntu or Alpine), which can evolve over time, making it difficult to
guarantee that a container built today will be identical in the future unless
explicitly managed through version pinning or using image digests
\cite{InotifySharedDrives}. Addressing this issue requires additional
effort in Docker-based deployments, where base images and package versions
must be manually pinned.
Containerization technologies are particularly effective in environments
where efficiency and scalability are crucial, such as
cloud-native applications and microservices architectures. Their ability
to provide lightweight, isolated environments with minimal overhead makes
them ideal for continuous integration and continuous delivery (CI/CD)
pipelines and other dynamic infrastructures \cite{Cgroups2024}.

\subsection{Architecture and Key Concepts}

Containerization systems are built on several key architectural components and
principles that work together to provide isolated, portable, and efficient environments
for applications. Understanding these components and their underlying mechanisms
highlights how containerization achieves its goals.

At the core of containerization is the concept of container images \cite{SystemImage2024}.
A container image is a lightweight, self-contained software package that includes
everything needed to run an application: code, runtime, libraries, and environment
settings. These images are built from declarative scripts, such as Dockerfiles, and
use Docker's layering system to improve efficiency \cite{UnderstandingImageLayers0200}.
Each instruction in the Dockerfile creates a new layer, and Docker employs caching
mechanisms to reuse unchanged layers during builds, significantly reducing image
build times \cite{Cache0200}.

Once built, these images are immutable, ensuring consistent and reproducible
environments across various systems. When executed, images are instantiated as
containers by a container runtime \cite{ContainerRuntimes}. The runtime is responsible
for managing the container lifecycle—creation, execution, and termination. Common
container runtimes include the Docker Engine \cite{DockerEngine0200}, containerd
\cite{Containerd}, and CRI-O \cite{Crio2024}. These runtimes leverage key features
of the underlying operating system, particularly namespaces and control groups
(cgroups), to provide isolation and resource control.

Namespaces are a Linux kernel feature that isolates different resources at the kernel
level. By using different types of namespaces, such as process ID (PID) \cite
{ProcessIdentifier2024}, network, and file system namespaces, containers have their
own isolated views of these system resources. For example, the PID namespace ensures
that each container has its own process tree, separate from others, while the network
namespace gives each container its own network stack \cite{LinuxNamespaces2024}.
Cgroups are another fundamental Linux kernel feature that ensures efficient resource
allocation for containers. Cgroups limit the amount of CPU \cite{CentralProcessingUnit2024},
memory, disk I/O \cite{InputOutput2024}, and other resources a container can use,
preventing any container from monopolizing system resources. This fine-grained control
helps maintain system stability and balance performance across containers
\cite{Cgroups2024}.
Linux Containers (LXC) represent an earlier container technology that also relies on
namespaces and cgroups. Linux containers run a full Linux distribution within the
container but require more manual configuration than Docker containers. Docker
containers, by contrast, emphasize ease of use and portability by packaging
applications and their dependencies in a standardized image format, simplifying
deployment across different environments \cite{LinuxContainers}.
Another critical component is container networking. Containers must communicate with
each other and with external systems. Containerization technologies provide several
networking models, such as bridge networks for inter-container communication on the
same host, and overlay networks for distributed applications across multiple hosts.
Host networking allows containers to share the host’s network stack directly
\cite{Networking0200}.
Storage is also vital in containerization. While containers are designed to be ephemeral,
many applications require persistent data. Containerization technologies offer solutions
such as volume mounts for persistent data storage, as well as networked or cloud-based
storage systems to ensure data integrity and persistence across container restarts and
migrations \cite{PersistingContainerData0200}.

By combining these architectural elements—container images, runtimes, namespaces,
cgroups, networking, and storage—containerization systems create a robust platform for
running applications in an isolated, portable, and efficient manner. The isolation
provided by namespaces and cgroups ensures that applications can run securely without
interfering with each other, while the immutability and layering of container images
enhance portability and efficiency. Together, these technologies have made
containerization a cornerstone of modern software development and deployment.

