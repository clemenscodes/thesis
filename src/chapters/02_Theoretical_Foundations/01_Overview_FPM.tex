\section{Functional Package Management}

This section introduces the fundamental concepts and scope of functional package
management, laying the groundwork for a detailed exploration of its core concepts,
architectural design, and practical examples.

\subsection{Definition and Scope}

Functional package management is a method for managing software packages by treating
them as immutable entities, ensuring reproducibility, and using declarative configuration
methods \cite[Section 2.1]{courtesFunctionalPackageManagement2013}. Unlike traditional
package management systems, which allow in-place upgrades and manual dependency handling,
functional package management guarantees that each package and its dependencies remain
unchanged once created \cite[Section 2.2]{courtesFunctionalPackageManagement2013}.
This immutability allows for the creation of reproducible environments, ensuring
consistent configurations across different systems and times, thus mitigating the
"it works on my machine" problem \cite[Section 2, Page 5]{rahmanWorksMeCannot2022}.

The scope of functional package management includes immutability, reproducibility,
and declarative configuration. Immutability means that any updates or changes to a
package result in a new version, leaving the original version intact
\cite[Page 84]{dolstraNixSafePolicyFree2004}. Reproducibility involves capturing
the exact state of the build environment, including all dependencies and their
versions, allowing for precise recreation of environments
\cite[Section 2.1]{courtesFunctionalPackageManagement2013}. Declarative configuration
involves describing the desired state of the system, with the package manager handling
the necessary adjustments to achieve that state \cite[Section 2.1]{courtesFunctionalPackageManagement2013}.

A core aspect of functional package management is dependency closure, which
ensures that all dependencies required for a package are completely defined and
isolated from the global system. This technique prevents dependency conflicts and
allows different versions of a package or its dependencies to coexist without
issues. It helps avoid "dependency hell" and ensures that packages are built in
isolation, so different applications or parts of the system can function independently
\cite[Chapter 3.3]{dolstraPurelyFunctionalSoftware2006}.

Functional package management systems also guarantee strict reproducibility by
hashing all inputs, including dependencies and system configurations used in the
build process. This ensures that package builds remain identical across different
systems and over time \cite[Section 2.2]{courtesFunctionalPackageManagement2013}.
By maintaining immutable package versions and using exact versions for every dependency,
these systems ensure that builds are not subject to drift over time.
This stands in contrast to traditional systems,
where updates to libraries and dependencies can introduce uncertainty into
future builds.

Furthermore, functional package management reduces system pollution. Developers can
use tools like \texttt{nix develop} or \texttt{nix-shell} to create isolated development
environments without modifying the global system. This approach allows quick
reinitialization by symlinking the required packages, significantly reducing
initialization time compared to container-based solutions
\cite{NixDevelopNix, NixshellNixReference}. As a result, developers can rapidly
switch between different project environments while ensuring that the global
environment remains clean and unaffected.

\subsection{Functional Package Management Systems}

Several systems exemplify the principles and benefits of functional package management.
These systems provide robust tools for managing software packages and their dependencies
in a reproducible and declarative manner, making them suitable for complex development
environments and deployment scenarios.

One prominent example is Nix, a functional package manager that employs a purely
functional approach to package management. Nix ensures that each build is isolated
and reproducible by using Nix expressions, which describe how packages are built.
This approach guarantees that every dependency is explicitly defined and accounted for,
allowing for precise replication of environments. Nix's unique features, such as the
Nix store, where packages are stored in a way that ensures immutability, and its garbage
collection mechanism, which automatically removes unused packages, make it a powerful
tool for managing software in a consistent and reliable manner
\cite[Section 2.2]{courtesFunctionalPackageManagement2013}.

Another notable example is GNU Guix, which is built on top of the Nix package manager
\cite{AcknowledgmentsGNUGuix} and extends its functionality with a focus on free
software principles. Guix uses Scheme, a dialect of Lisp \cite{LispProgrammingLanguage2024},
for its configuration language, providing a high degree of flexibility and power in
defining package configurations and system setups. Like Nix, Guix ensures reproducibility
and immutability, making it an excellent choice for environments where these properties
are critical.

Both Nix and Guix exemplify the core principles of functional package management.
They provide the tools and frameworks necessary to create and maintain reproducible,
immutable, and declaratively configured environments, offering significant benefits
for both development and deployment of software.

\subsection{Theoretical Concepts}

Functional package management systems are built on several key architectural components
and principles that work together to ensure immutability, reproducibility, and predictability
in software management. By understanding both the architecture and the underlying principles,
we gain insight into how these systems achieve their goals.

At the core of functional package management is the purely functional approach, where operations
like installing, updating, or removing a package do not have side effects. Instead of altering
the global system state, these operations create a new system state, ensuring predictability
and reversibility \cite[Section 2.1]{courtesFunctionalPackageManagement2013}. This concept
directly supports the architecture by ensuring system consistency even as packages are added
or modified.

A central component of the architecture is the Nix store, a centralized location where all
package versions and their dependencies are stored
\cite{guilloteauPainlessTranspositionReproducible2022}. Each package is stored under a unique
path based on its hash, ensuring that once a package is built, it cannot be modified, preserving
its immutability \cite[Chapter 5.2]{dolstraPurelyFunctionalSoftware2006}. This immutability
guarantees system stability, as packages cannot be altered once installed. Updates or
modifications result in new versions, which are stored alongside older versions, allowing
for versioning and rollback functionality \cite[Chapter 1.5]{dolstraPurelyFunctionalSoftware2006}.

Another critical architectural component is the build system, which constructs packages from
source code. The build system captures the complete environment required to build a package,
including all dependencies such as compiler versions and libraries. This ensures reproducibility,
meaning the same inputs will always produce the same outputs
\cite[Section 2.6]{dolstraPurelyFunctionalSoftware2006}. Reproducibility is crucial in ensuring
that software environments can be exactly recreated across different systems, an essential
feature for both development and production.

Functional package management also relies on declarative configuration, which simplifies system
management by allowing users to specify the desired state of the system instead of providing
step-by-step instructions on how to achieve it \cite[Section 2.2]{courtesFunctionalPackageManagement2013}.
This declarative approach is facilitated by the configuration language, a domain-specific language
or format used to declare package dependencies and system configurations. In Nix, this is
achieved through Nix expressions, which provide a flexible and powerful way to define system
configurations \cite[Chapter 2.2]{dolstraPurelyFunctionalSoftware2006}.

Dependency closure is another key concept underpinning functional package management. In traditional
systems, dependencies can conflict, causing issues across applications. In functional systems, each
package includes all its dependencies, ensuring independent functionality without conflicts
\cite[Chapter 3.3]{dolstraPurelyFunctionalSoftware2006}. This architectural decision ensures that
different versions of the same dependency can coexist without interfering with each other, a crucial
feature in complex environments.

Functional package management systems also employ garbage collection to manage disk space. Since
packages are immutable, old versions remain in the system unless explicitly removed. The system
automatically identifies and removes unused packages, ensuring that only the necessary packages
are retained \cite[Section 2.2]{courtesFunctionalPackageManagement2013}.

The combination of these architectural components and principles provides a robust and efficient
framework for managing software. The isolation of packages, made possible through dependency
closure and immutability, ensures that different packages do not interfere with each other. This
is particularly important in multi-application environments, where different applications may
require different versions of the same dependency \cite[Chapter 2.1]{dolstraPurelyFunctionalSoftware2006}.

By focusing on immutability, reproducibility, and declarative configuration, functional package
management systems offer significant advantages over traditional methods. The architecture is
designed to support these principles, providing predictability, reliability, and ease of use in
managing complex software environments.

Versioning and rollback capabilities provide a safety net by allowing users to revert to previous
system states, further enhancing system stability \cite[Chapter 1.5]{dolstraPurelyFunctionalSoftware2006}.
Each change in the system is transactional, ensuring that updates are consistent and can be reversed
if necessary.

Together, these key principles and architectural components form the foundation of functional
package management, ensuring that software environments are isolated, reproducible, and maintainable
across different use cases.
