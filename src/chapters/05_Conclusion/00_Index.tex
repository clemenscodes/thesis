\chapter{Conclusion}
This chapter provides a reflection on the findings of this thesis, which explored the 
comparative use of functional package management and containerization technologies in 
development environments and software deployment. Through systematic analysis and evaluation, 
the thesis has identified key strengths and limitations of each system, offering practical 
insights into their application in modern software development workflows. Additionally, 
this chapter presents suggestions for future research and discusses the broader implications 
of these findings for both practice and research.

\section{Summary of Findings}
This research focused on comparing functional package management and containerization as 
methods for managing development environments and software deployment. Both technologies 
show distinct strengths: functional package management excels in reproducibility and 
environment consistency, while containerization is more suited to scenarios requiring 
dynamic scaling and service orchestration.

In development environments, functional package management provides a significant advantage 
by allowing precise control over dependencies and ensuring reproducibility. By utilizing 
declarative configuration models, developers can define their environments with exact 
dependency versions, ensuring consistency across team members and eliminating common issues 
related to dependency conflicts or machine discrepancies. This capability simplifies the 
onboarding process for new developers, providing a seamless integration with the host system, 
especially on non-Linux platforms, where containers would require additional virtualization.

One key feature of functional package management is its use of binary caches, which allow 
developers to quickly retrieve pre-built binaries for dependencies. This reduces environment 
setup times considerably, especially in projects with large codebases and complex dependencies. 
Thus, functional package management is ideal for development workflows that prioritize 
efficiency and consistency, whether for individual developers or distributed teams working 
across diverse platforms.

In deployment scenarios, functional package management also demonstrates its effectiveness 
by enabling developers to bundle software binaries with their exact runtime dependencies. 
This ensures that applications run consistently across different systems, making it especially 
useful for self-hosted servers and personal desktops. By managing package versions precisely, 
functional package management reduces the risk of deployment errors due to incompatible 
runtime environments. Unlike containerization, which adds the overhead of virtualized 
environments, functional package management interacts directly with the host system, offering 
a lightweight solution for environments with stable traffic and minimal scalability needs.

By contrast, containerization shines in scenarios that require scalable, complex deployments, 
particularly for applications composed of multiple interdependent services. Containerization 
isolates each service in its own container, allowing them to run independently of the host 
system and other services. This isolation is particularly valuable when services have conflicting 
dependencies or different runtime environments. The ability to package each service separately 
makes containerization ideal for managing complex, service-oriented architectures.

Moreover, containerization integrates well with orchestration platforms such as Kubernetes, 
which automate the scaling and management of containers. These platforms dynamically monitor 
resource usage and traffic, adjusting the number of running containers to meet real-time 
application demands. This dynamic scalability makes containerization indispensable in 
environments with unpredictable traffic or applications that require frequent scaling. 
In summary, functional package management is best suited for maintaining reproducibility 
and control over both development and deployment environments, while containerization excels 
in handling large-scale, complex deployments that require dynamic service orchestration.

\section{Recommendations}
Based on these findings, several recommendations can be made for organizations deciding 
between functional package management and containerization technologies.

For development environments, functional package management is highly recommended over 
containerization. Its ability to provide fast, efficient, and reproducible environments 
makes it an excellent choice for development workflows. By leveraging declarative 
configurations and binary caches, teams can ensure consistent environments across developers, 
avoiding the dependency conflicts and setup challenges often encountered with containerized 
development environments. Containerization, while useful in deployment, adds unnecessary 
overhead for development, particularly where quick iterations, minimal resource usage, 
and tight integration with host systems are critical. Open-source projects would also benefit 
from adopting functional development environments, as contributors can easily set up 
consistent environments without extensive configuration.

For deployment, the choice between functional package management and containerization 
depends largely on the specific requirements of the application. Startups and small 
organizations should carefully evaluate whether containerization is necessary. In many 
cases, applications can be managed efficiently on self-hosted servers using functional 
package management without the added complexity of containers. This approach is ideal 
for stable, low-traffic environments where scalability is not a primary concern. However, 
if the application is expected to experience rapid growth or fluctuating traffic, containerized 
deployments may be more suitable, as they offer dynamic scaling to meet resource demands.

Containerized deployments should only be adopted when the application architecture justifies 
it—such as in cases involving complex microservices, rapid traffic changes, or where shutting 
down containers during off-peak hours saves resources. When integrated with orchestration 
platforms like Kubernetes, containerization provides the flexibility needed to manage such 
scenarios effectively. However, it should not be the default choice for deployments unless 
scalability requirements outweigh the simplicity and resource efficiency of functional 
package management.

\section{Future Work}
This thesis has identified several promising areas for future research, particularly in 
integrating the strengths of functional package management and containerization. One potential 
direction is the development of hybrid models that combine the reproducibility of functional 
package management with the scalability of containerized environments. Hybrid solutions could 
allow developers to define reproducible environments using functional package management, 
while leveraging container orchestration for dynamic scaling and service management. Research 
into how functional environments can be encapsulated within containers could help bridge the 
gap between these two approaches, enabling precise dependency control with the flexibility 
of container orchestration platforms.

Further research could explore methods for scaling functional package management in larger 
deployments. While functional package management is excellent for maintaining consistent 
environments, it currently lacks the dynamic scaling features found in containerized systems. 
By integrating functional package management with orchestration tools like Kubernetes, 
it may be possible to create systems that maintain strict reproducibility while offering 
real-time scaling. This could lead to functional package management becoming more viable 
for cloud-native deployments or environments where traffic demands fluctuate.

Additionally, improving the developer experience with functional package management through 
better tooling and automation could facilitate its adoption in larger teams and organizations. 
Developing tools that streamline service orchestration, enhance usability, and reduce the 
learning curve would help functional package management reach a broader audience. Such 
advancements could also enable its application in areas like infrastructure management, 
where predictable, reproducible environments are crucial.

\section{Implications for Practice and Research}
The findings of this thesis have important implications for both practical applications 
and academic research. Practically, these insights can help guide organizations in selecting 
the appropriate technology based on their specific needs. For development environments, 
functional package management offers superior reproducibility and efficiency, making it 
a better choice than containerization. Organizations should prioritize using functional 
package management to eliminate dependency conflicts and speed up environment setup. Open-source 
projects, in particular, can benefit from the consistency and ease of use provided by 
functional development environments.

For deployment workflows, containerization remains the best option in scenarios that require 
scalability, service orchestration, and flexibility. However, startups and small organizations 
should consider whether containerization is truly necessary or if functional package management 
can provide a simpler, more resource-efficient solution. In cases where traffic is predictable 
and scalability is not a primary concern, functional package management should be the default 
choice for deployment.

From an academic perspective, this research opens new avenues for exploring hybrid models 
that combine the benefits of both functional package management and containerization. Future 
studies could focus on how functional environments can be deployed within containers, 
maintaining reproducibility while allowing for dynamic scaling. Additionally, research 
into improving the scalability and orchestration capabilities of functional package management 
could lead to new frameworks for managing large-scale distributed systems. Comparative studies 
on the long-term operational efficiency, cost, and security of both systems would provide 
further insights for practitioners, helping them make more informed decisions when selecting 
the most appropriate technology for their use cases.
