\chapter{Conclusion}
This chapter provides a comprehensive reflection on the findings of this thesis, which
explored the comparative use of functional package management and containerization
technologies in development environments and software deployment. By systematically
analyzing and evaluating the performance of Nix and Docker, this thesis has identified
key strengths and limitations of each technology, offering practical recommendations for
their usage in modern software development workflows. Additionally, the chapter presents
suggestions for future work and discusses the broader implications of these findings
for both practice and research.

\section{Summary of Findings}
The primary research question of this thesis aimed to compare Nix and Docker as tools
for managing development environments and software deployment. Through detailed
analysis and empirical evaluation, it became evident that both tools excel in different
areas and cater to distinct use cases.

In development environments, Nix provides a highly effective solution for setting up
and maintaining development environments, offering reproducibility and dependency
management that is unmatched by Docker. By using \texttt{nix-shell}, developers can
instantly create isolated development environments without the need for containers or
virtual machines. This reduces the overhead associated with container-based environments,
especially on non-Linux systems where Docker requires a virtual machine to emulate
Linux. Nix also benefits from binary caches, allowing developers to pull pre-built
packages and environments from shared caches, significantly speeding up setup times.
This makes Nix ideal for both individual developers and teams, as the time saved by
eliminating lengthy build processes leads to greater productivity.

The findings suggest that Nix should be recommended for any development environment,
regardless of scale. For smaller teams or startups, the ability to provide quick, consistent,
and reproducible environments is particularly beneficial, reducing the friction typically
associated with onboarding new developers or working across different machines. The
cost-efficiency and ease of use further make Nix an attractive option for companies that
do not require the overhead of managing containers for development.

In the context of software deployment, the choice between Nix and Docker depends largely
on the scale and traffic demands of the application. Nix is well-suited for smaller-scale,
self-hosted solutions where traffic is relatively stable and does not fluctuate significantly.
For startups or small companies that may not expect major traffic spikes, deploying
applications on rented virtual private servers using Nix can be a highly cost-effective
solution. The ability to manage the entire environment declaratively with Nix, combined
with its package isolation, ensures a stable and reproducible deployment process. However,
Nix's requirement for a fully functioning server with the Nix package manager can introduce
complexity for larger-scale applications that require more flexibility and scalability.

For high-traffic applications and large-scale companies, Docker's containerization
capabilities offer clear advantages. Docker excels in dynamic, scalable environments,
where applications must be able to handle fluctuating traffic demands. Docker's
integration with orchestration platforms like Kubernetes allows for automated scaling,
making it the preferred choice for microservices architectures and serverless deployments.
Docker containers are lightweight, portable, and can be easily managed across different
infrastructure platforms, giving organizations the ability to scale services up or down
with minimal effort.

Despite Docker's superiority in handling large-scale deployments, the use of Nix for
development in such environments should not be overlooked. Even in large-scale
organizations, utilizing Nix for development environments can provide substantial
benefits. Nix's ability to maintain reproducible development environments, combined
with its efficient caching mechanisms, ensures that developers spend less time setting
up environments and more time coding. By caching developer shells and reusing
dependencies, Nix significantly reduces friction in the development process, particularly
when iterating on complex codebases. Therefore, even for large-scale projects that use
Docker for deployment, Nix remains a highly valuable tool for development, offering
time-saving benefits that improve overall productivity.

\section{Recommendations}
Based on the findings of this research, several practical recommendations can be made
for developers and organizations when selecting between Docker and Nix for different
use cases. For small companies and startups, particularly those that are self-hosting
or working with rented virtual private servers (VPS), adopting Nix for both development
and deployment is highly recommended. Nix provides a lightweight and reproducible
approach to managing environments, reducing the operational overhead often associated
with containerization technologies. Since Nix declaratively manages package dependencies
and builds reproducible environments, it is particularly suitable for teams that do not
need dynamic scaling but still require consistency across development and production
environments. This approach works well for environments where traffic is stable, and
scaling requirements are minimal. Nix allows smaller organizations to focus on efficient
use of resources and maintain a streamlined development process without dealing with the
complexity of container orchestration or dynamic scaling.

For companies working in environments with more dynamic traffic or high scalability
requirements, Docker remains the most practical option for handling deployments. Docker’s
containerization technology, especially when paired with orchestration tools like
Kubernetes, allows applications to scale seamlessly based on demand. Containers
are lightweight, portable, and can be spun up and shut down quickly, providing
resource efficiency and reducing downtime in high-traffic applications. The flexibility
Docker offers in terms of scaling services independently is crucial for organizations
that operate at larger scales or employ microservice architectures. The ability to
efficiently manage application scaling through tools like Kubernetes is particularly
important for large-scale companies that need to dynamically allocate resources to
different services as traffic changes.

Despite Docker’s advantages in production deployment for larger applications, Nix
remains a powerful tool for managing development environments, even in large
organizations. By leveraging Nix to manage development environments, teams can
benefit from fast, reproducible setups that eliminate the friction often associated
with configuring environments manually or dealing with dependency conflicts. Nix’s
ability to cache environments and reuse previously built dependencies means that
developers spend less time setting up environments and more time coding. This is
particularly valuable in larger teams where consistent environments are required
across many developers. The combination of Nix for development and Docker for deployment
provides a hybrid solution where Nix guarantees reproducibility during development
while Docker offers scalability and flexibility in production.

\section{Future Work}
The research conducted in this thesis opens up several avenues for further investigation
and development. One promising area of future research could focus on optimizing the
integration between Nix and container-based deployment models. While Docker is highly
effective at handling process isolation and scalability, Nix’s functional package
management could be combined with containerization to create hybrid solutions that
leverage the strengths of both tools. For instance, a future direction might explore
how Nix environments can be deployed within containers to ensure both reproducibility
and scalability. This would allow organizations to benefit from the rapid environment
setup Nix offers, while also taking advantage of Docker’s scalability for production
deployments.

Another area worth exploring is improving the scalability of Nix in high-traffic
environments. Currently, while Nix excels at package management and reproducibility,
it lacks the dynamic scaling capabilities offered by Docker and Kubernetes. Further
research could investigate how Nix deployments might be adapted to scale more effectively,
potentially through the use of orchestration tools specifically tailored to Nix’s package
management model. Additionally, developing tools or extensions that make Nix more suitable
for dynamic traffic environments could make it a viable option for larger-scale
deployments that require the flexibility to adjust resource usage in real-time.

Another avenue for future work could explore improving developer experience and
onboarding with Nix in larger organizations. Although tools like \textbf{services-flake}
show promise, further work could be done to simplify process orchestration and make
it easier for larger teams to adopt Nix. By enhancing the usability of Nix in more
complex environments, it could become an even stronger competitor to Docker for
managing both development and production environments.

\section{Implications for Practice and Research}
The findings of this thesis have significant implications for both practical applications
and academic research. From a practical perspective, organizations now have clearer
guidelines for when to use Nix or Docker based on their project requirements. Small
companies and startups that focus on maintaining consistency across environments without
needing to scale dynamically will find Nix highly effective for both development and
deployment. By contrast, organizations that require rapid scaling and process isolation
will benefit more from Docker’s containerization model, especially when deploying
high-traffic applications using orchestration tools like Kubernetes.

The research also highlights the potential value of using Nix for development, even in
environments where Docker is preferred for production. The use of Nix in development
environments can reduce the overhead and friction associated with setting up and
maintaining development environments, particularly in large organizations. This hybrid
approach, where Nix is used for development and Docker for deployment, ensures that
teams can maintain reproducibility and efficiency across the software lifecycle,
balancing the strengths of both technologies.

For academic research, this thesis contributes to the growing body of knowledge on
software deployment and package management. It provides a comparative analysis of
functional package management and containerization, identifying where each approach
excels and where improvements can be made. Future research could build on these
findings by exploring how Nix and Docker can be integrated more effectively or how
the scalability challenges of Nix could be addressed. There is also an opportunity
to study how the principles of functional package management could be applied to
other areas of software development and infrastructure management, potentially
leading to new hybrid models that balance the benefits of reproducibility,
scalability, and flexibility.
