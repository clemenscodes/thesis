\section{Continuous Integration and Continuous Deployment}

Continuous Integration and Continuous Deployment are key practices in modern
software engineering. These practices automate the processes of building, testing,
and deploying applications, ensuring consistent, reliable workflows. For the
application developed in this thesis, GitHub Actions was used to orchestrate the
CI/CD pipeline. GitHub Actions is particularly suited for CI/CD pipelines because
of its seamless integration with GitHub repositories, especially for projects
hosted on GitHub.

One of the main benefits of GitHub Actions is its ability to trigger workflows
automatically based on events like \texttt{push} or \texttt{pull\_request}. This
automation ensures that new code is built, tested, and deployed immediately,
without manual intervention. Moreover, GitHub Actions is free for public
repositories, which makes it a cost-effective solution for open-source projects
and smaller teams. In this project, GitHub Actions handles the entire CI/CD process
without additional infrastructure costs.

GitHub Actions also provides a variety of pre-built actions and plugins that make
it easy to set up and manage complex workflows. These actions can handle tasks such
as checking out the repository, logging into container registries, setting up Docker
environments, and even managing multi-platform builds. In this application, several
pre-built actions are used to automate the CI/CD process, reducing the need for
custom scripts. This increases both the speed and reliability of the pipeline,
since these actions are maintained by a broader community and official providers.

A key feature of GitHub Actions is its built-in caching mechanism, which helps
improve performance for containerized applications. For projects using Docker,
caching can significantly reduce build times by avoiding the need to rebuild all
layers of a Docker image from scratch. Instead, unchanged layers from previous
builds are reused. This optimization is especially useful in large projects where
Docker images may consist of multiple layers. In the application, Docker layer
caching is employed extensively to speed up the CI/CD pipeline, ensuring efficient
and rapid builds.
The workflow builds and pushes Docker images to the GitHub Container Registry
(GHCR). GHCR is tightly integrated with GitHub, offering a secure and convenient
place to store Docker images. GitHub Actions can log into GHCR using tokens and
permissions, avoiding the need for external credentials. This integration ensures
secure image management, reduces friction when working with container registries,
and enhances the overall security model by avoiding the need to manage sensitive
credentials manually.

The CI/CD pipeline is composed of several steps, each designed to handle specific
parts of the process. Below are the key sections of the workflow file, along with
explanations of the important actions.

The first step defines the events that trigger the pipeline. In this case, the
workflow is triggered when code is pushed to the \texttt{main} branch or when a
pull request is created or updated.

\begin{lstlisting}[caption={Triggering Events}]
name: Docker GHA Cache
on:
  push:
    branches: [main]
  pull_request:
    types: [opened, synchronize]
\end{lstlisting}

This setup ensures that any code changes pushed to the \texttt{main} branch, or any
modifications to an open pull request, will trigger the CI/CD pipeline. This helps
maintain continuous feedback and ensures that code is always tested and built
consistently.

Concurrency control is also defined to prevent multiple workflows from running
simultaneously for the same branch or pull request. This avoids redundant builds and
ensures that only the most recent changes are processed.

\begin{lstlisting}[caption={Concurrency Settings}]
concurrency:
  group: ${{ github.workflow }}-${{ github.event.number || github.ref }}
  cancel-in-progress: true
\end{lstlisting}

Environment variables are used to define key settings like the Docker registry and
image name. By using environment variables, the workflow becomes more flexible and
easier to maintain.

\begin{lstlisting}[caption={Environment Variables}]
env:
  DOCKER_REGISTRY: ghcr.io
  IMAGE_NAME: ${{ github.repository }}
\end{lstlisting}

This setup ensures that the Docker image is built with the correct registry and
repository name, avoiding the need to hardcode these values in the workflow.

The next part of the workflow defines the job that will run on an
\texttt{ubuntu-latest} runner. The repository is checked out using the
\texttt{actions/checkout@v4} action to ensure that the latest code is available for
the build process.

\begin{lstlisting}[caption={Job Setup and Repository Checkout}]
jobs:
  docker:
    runs-on: ubuntu-latest
    permissions:
      id-token: write
      contents: read
      packages: write
    steps:
      - name: Checkout repository
        uses: actions/checkout@v4
\end{lstlisting}

Once the code has been checked out, the workflow logs into the GitHub Container
Registry using the \texttt{docker/login-action@v3}. This action allows GitHub
Actions to securely authenticate and push Docker images to the registry using
\texttt{GITHUB\_TOKEN}.

\begin{lstlisting}[caption={Logging into the GitHub Container Registry}]
- name: Log in to the Container registry
  uses: docker/login-action@v3
  with:
    registry: ${{ env.DOCKER_REGISTRY }}
    username: ${{ github.actor }}
    password: ${{ secrets.GITHUB_TOKEN }}
\end{lstlisting}

Using the \texttt{docker/setup-buildx-action@v3} action, Docker Buildx is set up.
Buildx provides advanced features such as
multi-platform builds and more efficient caching.

\begin{lstlisting}[caption={Setting up Docker Buildx}]
- name: Set up Docker Buildx
  uses: docker/setup-buildx-action@v3
\end{lstlisting}

Metadata for the Docker image is extracted using the
\texttt{docker/metadata-action@v5}. This ensures that the image is properly tagged
and labeled for version management.

\begin{lstlisting}[caption={Extracting Metadata for Docker Image}]
- name: Extract metadata (tags, labels) for Docker
  id: meta
  uses: docker/metadata-action@v5
  with:
    images: ${{ env.DOCKER_REGISTRY }}/${{ env.IMAGE_NAME }}
\end{lstlisting}

Finally, the Docker image is built and pushed to the GitHub Container Registry using
the \texttt{docker/build-push-action@v6}. Caching is enabled to reuse unchanged
layers from previous builds, significantly reducing build times.

\begin{lstlisting}[caption={Building and Pushing the Docker Image}]
- name: Build and push Docker image
  id: push
  uses: docker/build-push-action@v6
  with:
    context: .
    push: true
    tags: ${{ steps.meta.outputs.tags }}
    labels: ${{ steps.meta.outputs.labels }}
    cache-from: type=gha
    cache-to: type=gha,mode=max
\end{lstlisting}

This caching mechanism is vital for optimizing the CI/CD pipeline, as it speeds up
subsequent builds by avoiding the need to rebuild Docker layers that have not
changed. This is particularly useful for larger projects where Docker images
consist of multiple layers.

For the Nix-based project components, two different CI pipelines are used: one that
leverages GitHub Actions' built-in caching and another that uses Cachix, a service
designed to cache Nix build artifacts across different environments. Both pipelines
help ensure reproducibility by caching build results and avoiding unnecessary
rebuilds.

The first workflow uses the \texttt{magic-nix-cache} action to store Nix build
artifacts within GitHub Actions’ cache. This setup is fully integrated within the
GitHub environment, making it easy to use.
The job runs on an \texttt{ubuntu-latest} runner, and Nix is installed using the
\texttt{DeterminateSystems/nix-installer-action@main}. The code is checked out, and
the build process is initiated using the Nix flake configuration.

\begin{lstlisting}[caption={GitHub Runner and Nix Installation}]
jobs:
  magic-nix-cache:
    runs-on: ubuntu-latest
    permissions:
      id-token: "write"
      contents: "read"
    steps:
      - uses: actions/checkout@v4
      - uses: DeterminateSystems/nix-installer-action@main
      - uses: DeterminateSystems/magic-nix-cache-action@main
\end{lstlisting}

Cachix is used in the second pipeline to provide caching that can be shared between
CI pipelines and local developer environments. This allows developers to reuse
build results across environments, improving productivity.
The key element in the Nix CI pipeline using Cachix is that it allows caching
of Nix flake inputs, development shells, and runtime closures. By pushing
these caches to the Cachix service, developers can reuse previously built
derivations, improving both CI and local development build times.

\begin{lstlisting}[caption={Caching Nix Flake Inputs}]
      - name: Cache flake inputs
        run: |
          nix flake archive --json \
            | jq -r '.path,(.inputs|to_entries[].value.path)' \
            | cachix push ${{ env.NIX_CACHE }}
\end{lstlisting}

The `nix flake archive` command is used to archive Nix flake inputs (dependencies
defined in the Nix configuration), and the results are pushed to the Cachix cache
to be reused in future builds.

The next step caches the development shell, enabling developers to pull a cached
shell environment, ensuring that their local environment is identical to the CI
environment.

\begin{lstlisting}[caption={Caching the Development Shell}]
      - name: Cache development shell
        run: |
          nix develop --accept-flake-config --profile \
            ${{ env.NIX_DEV_PROFILE }} -c true
          cachix push ${{ env.NIX_CACHE }} ${{ env.NIX_DEV_PROFILE }}
\end{lstlisting}

Finally, the runtime closures (outputs of the Nix build) are cached. This ensures
that the project can be rebuilt without needing to recompile unchanged parts of
the project, saving time.

\begin{lstlisting}[caption={Caching Runtime Closures}]
      - name: Cache runtime closures
        run: |
          nix build --accept-flake-config --json \
            | jq -r '.[].outputs | to_entries[].value' \
            | cachix push ${{ env.NIX_CACHE }}
\end{lstlisting}

By caching both the flake inputs and the development shell, the Nix CI pipeline
significantly reduces build times for both CI and local environments, making it
a robust and efficient system for handling functional package management in the
project.

This chapter has provided an in-depth exploration of the CI/CD pipeline setup for
the application, focusing on two key technologies: containerization using Docker
and functional package management through Nix. The workflows designed for the
project not only ensure reproducibility and consistency but also streamline the
development process by automating key tasks such as building, testing, and deploying
the application. These workflows, powered by GitHub Actions, emphasize best practices
for modern software engineering, such as dependency version pinning, caching, and
secure image management.

The use of Docker and Nix in the CI/CD pipeline demonstrates how containerization
and functional package management can be employed to maintain a stable development
environment, reducing the risk of inconsistencies between development, testing,
and production environments. Caching mechanisms, both for Docker images and Nix
derivations, play a crucial role in speeding up the build process, making the entire
CI/CD pipeline more efficient.

The workflow designs provided here are not specific to any
particular software stack or industry but represent best practices that can be
adopted across the board to ensure consistency, security, and performance in
development operations.

It is important to note that everything described in this chapter serves as both
a guideline and a proof of concept for how companies can implement reproducible,
efficient CI/CD pipelines utilizing functional package management
or containerization technologies in their own environments.
The tools and concepts discussed
here are widely applicable and can be adapted to fit various types of projects and
software architectures.

