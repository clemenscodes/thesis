\section{Reproducible Builds}

In this section, we explore how reproducible builds are achieved
using functional package management
through a Nix-based approach, focusing on concepts such as dependency isolation,
deterministic builds, and environment specification.
Reproducible builds are an essential concept in software development that ensures
anyone can build the exact same binary from the same source code, across different
machines and environments. The idea is to eliminate any potential for variance
between builds, such as those caused by differing system configurations, dependencies,
or build tools. In a reproducible build, the input always results in the same output.
This ensures that developers can trust the integrity and consistency of their builds,
regardless of where or when the build process takes place.

\subsection{Reproducible Builds Using Functional Package Management}

Functional package management systems like Nix introduce immutability and isolation
into the package management process. Every environment is built from a set of
declarative rules, ensuring that the same rules always yield the same result.
The key to reproducible builds in functional package management lies in the
declarative definition of dependencies and build processes, which eliminates
the impact of external factors like system state or environment configuration.
The build process in Nix relies on expressions, such as those written in the
\texttt{default.nix} file, which define how the software should be built and what
dependencies are required. These expressions are purely functional, meaning that
their output is determined entirely by the input. Let us explore how the Nix
expression for building a Rust application ensures reproducibility.
The \texttt{default.nix} file begins by importing and filtering the source files,
which is crucial for maintaining a well-defined input set. The source code and
other relevant files are explicitly listed, ensuring that only these files are
used during the build process. This isolates the build from any extraneous files
in the directory:

\begin{lstlisting}[caption={Source filtering in Nix}]
src = filter {
  root = ./.;
  include = [
    ./src
    ./styles
    ./templates
    ./assets
    ./Cargo.lock
    ./Cargo.toml
  ];
};
\end{lstlisting}

This approach ensures that only the specified files—such as the \texttt{src} directory,
stylesheets, templates, and the \texttt{Cargo.toml} and \texttt{Cargo.lock} files—are
included in the build. By filtering the input, Nix guarantees that no unintended
files or external dependencies are introduced into the build process, contributing
to the reproducibility of the final output.
It also ensures that a new derivation is only built
if the source files for the application changed.

Another key element of reproducible builds is dependency management. In Rust
projects, the \texttt{Cargo.lock} file locks the exact versions of dependencies,
ensuring that the same versions are used every time the project is built. Nix
leverages this concept by importing and hashing the dependencies from the
\texttt{Cargo.lock} file:

\begin{lstlisting}[caption={Rust dependency management in Nix}]
cargoDeps = rustPlatform.importCargoLock {
  lockFile = ./Cargo.lock;
};
cargoHash = "sha256-EYTuVD1SSk3q4UWBo+736Mby4nFZWFCim3MS9YBsrLc=";
\end{lstlisting}

Here, the \texttt{importCargoLock} function imports the exact dependency versions
defined in \texttt{Cargo.lock}. Additionally, the \texttt{cargoHash} attribute ensures
that the same dependencies are used by verifying the integrity of the \texttt{Cargo.lock}
file. This hashing mechanism ensures that even if the dependencies were to change
in the source repository, the build process would fail unless the \texttt{Cargo.lock}
file remains unchanged, thereby guaranteeing reproducibility.

In functional package management systems, builds are isolated from the host system,
and all dependencies must be explicitly defined. This prevents any contamination
from system libraries or tools, which can lead to non-reproducible builds. In the
Nix expression, the \texttt{nativeBuildInputs} and \texttt{buildInputs} attributes
explicitly define the dependencies needed for building the application:

\begin{lstlisting}[caption={Defining build dependencies in Nix}]
nativeBuildInputs = [pkg-config];
buildInputs = [openssl];
\end{lstlisting}

By listing the necessary build inputs (such as \texttt{pkg-config} and \texttt{openssl}),
Nix ensures that the build process does not rely on any system-wide installed tools
or libraries. Instead, all dependencies are fetched from the Nix package store,
where they are versioned and managed immutably. This further strengthens the
reproducibility of the build, as the same versions of these dependencies will be
used regardless of the machine or environment in which the build is performed.

Once the application is built, a script can be defined to run the application
with the necessary environment variables. This is achieved using the
\texttt{writeShellScriptBin} function, which ensures that the correct paths
are set for the application’s runtime assets:

\begin{lstlisting}[caption={Generating the run script in Nix}]
writeShellScriptBin pname ''
  WEBSERVER_ASSETS=${assets}/assets ${unwrapped}/bin/webserver
''
\end{lstlisting}

This script ensures that the \texttt{assets} directory and other runtime resources
are correctly set up when the application is executed. Since all dependencies,
build tools, and runtime configurations are explicitly defined within the Nix
expression, the build and execution of the application remain reproducible
regardless of the underlying environment.

\subsection{Reproducible Builds Using Containerization Technologies}

Containerization technologies, such as Docker, are key to achieving reproducible builds
by isolating the entire build environment from the host system. A Dockerfile defines
a controlled environment where the exact dependencies, system libraries, and tools
are specified, ensuring consistency across different machines and platforms. By
following best practices, Docker enables reproducible builds that maintain the integrity
of the software, regardless of external factors such as system configuration or host
environment changes.

The Dockerfile used in this project adheres to several best practices, including
multi-stage builds, dependency version pinning, and build optimization. Each section
of the Dockerfile is crafted to ensure that the resulting binary is reproducible and
that the build process remains efficient.

The Dockerfile begins by specifying a base image, \texttt{rust:1.80.1-slim-bullseye},
which provides a minimal Debian environment with the Rust toolchain pre-installed.
Pinning the image version ensures that the same version of Rust is used in every build,
eliminating variability that could arise from different versions of the language or
runtime. This helps achieve reproducibility since it guarantees that the environment
remains the same over time.

\begin{lstlisting}[caption={Base Image and Environment Setup}]
FROM rust:1.80.1-slim-bullseye AS base

ENV DEBIAN_FRONTEND=noninteractive
ENV SHELL=/bin/bash
ENV PATH="/root/.proto/bin:$PATH"
\end{lstlisting}

\begin{lstlisting}[caption={Installing Dependencies in Docker}]
RUN apt-get update && \
  apt-get install -y --no-install-recommends \
  git=1:2.30.2-1* \
  gzip=1.10-4* \
  unzip=6.0-26* \
  xz-utils=5.2.5-2.1* \
  curl=7.74.0-1.3* \
  pkg-config=0.29.2-1* \
  openssl=1.1.1* \
  libssl-dev=1.1.1* \
  musl-tools=1.2.2-1* \
  make=4.3-4.1* \
  && \
  apt-get clean && \
  rm -rf /var/lib/apt/lists/*
\end{lstlisting}

In this section, system dependencies are installed using the \texttt{apt-get} package
manager. Each package is pinned to a specific version, which ensures that future builds
use the exact same versions, preventing inconsistencies caused by upstream changes.
The \texttt{--no-install-recommends} option is used to avoid installing unnecessary
packages, keeping the image as small as possible.

After installation, \texttt{apt-get clean} is run to free up disk space,
reducing the size of the Docker image. Additionally,
the \texttt{SHELL} directive with \texttt{-o pipefail} ensures that if any command in
a pipeline fails, the entire build will fail, which prevents silent failures during
the build process.

The Dockerfile also includes the use of multi-stage builds to keep the final image
small and efficient. Multi-stage builds allow us to use a larger image with necessary
build tools for compiling the application, and then copy only the compiled binary and
essential files into a smaller runtime image.

\begin{lstlisting}[caption={Multi-Stage Build Setup}]
FROM base AS build

COPY Cargo.toml Cargo.lock ./
RUN rustup target add x86_64-unknown-linux-musl && \
    cargo build --release --target=x86_64-unknown-linux-musl

FROM alpine:3.20.2 AS start

COPY --from=build /app/webserver /usr/local/bin/webserver
\end{lstlisting}

The first stage, labeled \texttt{build}, compiles the Rust application. The
\texttt{Cargo.toml} and \texttt{Cargo.lock} files are copied into the container to
resolve dependencies. By copying these files early, Docker can cache the dependency
resolution step, improving build times by ensuring that dependencies are only
recompiled if they change. The Rust target is set to \texttt{x86\_64-unknown-linux-musl},
ensuring that the binary is statically linked, making it more portable across different
environments.

The final stage, labeled \texttt{start}, uses the minimal \texttt{alpine:3.20.2} image.
Only the compiled binary and necessary runtime files are copied from the \texttt{build}
stage into this smaller image. This keeps the final image lightweight by excluding
the build tools and unnecessary libraries from the runtime environment. This separation
of build and runtime environments is a best practice that results in a secure and
efficient Docker image.

To ensure the Rust binary is fully self-contained and does not rely on dynamic libraries,
the Dockerfile compiles OpenSSL against the MUSL C library, resulting in a statically
linked binary. This approach improves portability, as the binary can be deployed on any
system without requiring shared libraries such as OpenSSL.

\begin{lstlisting}[caption={OpenSSL Compilation with MUSL}]
RUN ln -s /usr/include/x86_64-linux-gnu/asm /usr/include/x86_64-linux-musl/asm && \
    ln -s /usr/include/asm-generic /usr/include/x86_64-linux-musl/asm-generic && \
    ln -s /usr/include/linux /usr/include/x86_64-linux-musl/linux && \
    mkdir /musl && \
    curl -LO https://github.com/openssl/openssl/archive/OpenSSL_1_1_1f.tar.gz && \
    tar zxvf OpenSSL_1_1_1f.tar.gz

WORKDIR /openssl/openssl-OpenSSL_1_1_1f/
RUN CC="musl-gcc -fPIE -pie" ./Configure no-shared no-async --prefix=/musl \
    --openssldir=/musl/ssl linux-x86_64 && \
    make depend && \
    make -j"$(nproc)" && \
    make install
\end{lstlisting}

This process statically links OpenSSL with the Rust binary, enabling the Rust
application to handle HTTPS connections without needing shared OpenSSL libraries at
runtime. By statically linking critical libraries, the resulting binary becomes
fully portable and can be executed in a minimal container, such as the Alpine image,
without additional dependencies.
Security best practices are also implemented by running the application as a non-root
user, which reduces the risk of privilege escalation attacks. In the final stage,
a non-root user named \texttt{webserver} is created, and the ownership of the binary
is transferred to this user.

\begin{lstlisting}[caption={Running the Application as a Non-Root User}]
RUN addgroup -g 1000 webserver && \
    adduser -D -s /bin/sh -u 1000 -G webserver webserver && \
    chown webserver:webserver /usr/local/bin/webserver

USER webserver

CMD ["webserver"]
\end{lstlisting}

Running the application as a non-root user is a Docker best practice, as it limits
the potential impact of any security vulnerabilities within the application. The
\texttt{CMD} directive ensures that the container will execute the \texttt{webserver}
binary when it starts.

This Dockerfile follows several best practices to ensure a reproducible, secure, and
efficient build of the Rust web server. By using multi-stage builds, pinning dependency
versions, and statically linking critical libraries, the Dockerfile guarantees that
the build is consistent and portable across different systems. These practices also
ensure that the final image is as small as possible, minimizing the attack surface and
reducing resource usage. The use of a non-root user and the minimal Alpine image
further enhance the security and efficiency of the resulting container, making it
well-suited for deployment in production environments.

