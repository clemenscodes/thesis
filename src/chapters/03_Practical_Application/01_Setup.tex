\section{Setup}

Setting up a reproducible development environment is crucial for ensuring consistency
and reliability in software development, especially when working across various
platforms and with diverse team members. In this section, we focus on two
specific tools: Nix and Docker, which serve as examples of functional package
management and containerization technologies, respectively. Both tools play a
vital role in modern development workflows, ensuring that development
environments can be recreated accurately, no matter where they are deployed.

The following command installs Nix as a multi-user system
on Unix-based systems such as Linux and macOS.

\begin{lstlisting}[caption={Installing Nix on Linux and macOS}]
sh <(curl -L https://nixos.org/nix/install) --daemon
\end{lstlisting}

After installation, Nix must be integrated into the user's shell environment to
make its commands available. This is done by sourcing the Nix configuration file,
which is automatically generated during installation. This step ensures that the `nix`
command and its associated functionalities are accessible within the current shell session.

\begin{lstlisting}[caption={Configuring Shell for Nix}]
source /nix/var/nix/profiles/default/etc/profile.d/nix-daemon.sh
\end{lstlisting}

For Windows users, Nix is not natively supported but can be run using the Windows
Subsystem for Linux (WSL). WSL provides a Unix-like environment on Windows systems,
allowing users to install and run Linux applications natively. The same installation
procedure used on Linux applies within the WSL environment, making it possible for
Windows users to leverage the functional package management capabilities of Nix.

Once installed and configured, Nix allows users to define their entire development
environment in a single configuration file (e.g., `default.nix`). This file specifies
the exact versions of all required dependencies, ensuring that the environment can
be reproduced across different machines. The ability to roll back environments to
previous states also helps mitigate issues that might arise during software updates.

To install Docker on Linux,
users can utilize their system’s package manager. For Debian-based systems,
the following commands will install Docker and set up permissions so that users can
run Docker without needing superuser privileges:

\begin{lstlisting}[caption={Installing Docker on Linux}]
sudo apt-get update
sudo apt-get install docker-ce docker-ce-cli containerd.io
sudo usermod -aG docker $USER
newgrp docker
\end{lstlisting}

For Windows users, Docker Desktop is available, which uses WSL2 as the backend to
run containers. After installing Docker Desktop, users can interact with Docker
via its GUI or command-line tools. Docker Desktop also
handles system-level configurations for networking and storage, making it easier
to manage containers on Windows systems.

In summary, the setup of development environments using Nix and Docker ensures
that both package-level and runtime dependencies can be managed in a reproducible
and isolated manner. These tools form the foundation of reproducible development
workflows, ensuring that development environments can be easily recreated across
different systems and stages of software development.
