\section{Objectives}

The primary goal of this thesis is to provide a comprehensive comparison between
functional package management and containerization technologies as development and
deployment tools. The thesis seeks to analyze these two paradigms through empirical
and practical exploration, focusing on key metrics such as build times, deployment
speeds, package sizes, and the management of developer environments.

At the heart of this research is the central question: \textbf{How do functional package
	management and containerization technologies compare as development and deployment
	tools, and what recommendations can be derived for enterprises that use these
	technologies?} The purpose of this research is to offer insights that help developers
and organizations make informed decisions about which technology to adopt depending
on their specific needs.

In order to thoroughly explore this question, the thesis pursues several objectives.
First, it aims to examine the philosophical underpinnings of both functional package
management and containerization. Functional package management emphasizes immutability,
reproducibility, and declarative configurations. Containerization technologies, on
the other hand, center on process isolation, portability, and flexibility. These core
principles are vital to understanding the broader context in which each paradigm
excels. It is essential to investigate how these philosophies impact the setup and
management of development environments, software reproducibility, and the overall
software lifecycle.

The second objective is to conduct a technical analysis of the implementation details
that differentiate these technologies. By evaluating their declarative configurations
and lifecycle management, the thesis explores how these approaches provide strengths
and expose limitations, particularly in areas like environment isolation, package
dependencies, and configuration complexity. These technical distinctions reveal the
potential benefits and constraints of using either functional package management or
containerization technologies in different development contexts.

The third objective is to apply these technologies to a real-world project to provide
empirical evidence of their effectiveness. A web server written in Rust serves as the
test bed for demonstrating how functional package management and containerization
behave under practical conditions, with a focus on dependency management, environment
isolation, build reproducibility, and deployment. This case study illustrates how each
technology can handle typical tasks faced by development teams and highlights specific
scenarios where one approach may be more suitable than the other.

Finally, the thesis evaluates performance metrics such as build times, package sizes,
and deployment scalability. Build times are crucial because they directly affect
developer productivity and the continuous integration/continuous deployment (CI/CD)
pipeline. By comparing containerization's layered caching approach with the derivation-based
build system of functional package management, the research examines how quickly each
system can produce artifacts under various conditions. Package size, another key metric,
has significant implications for storage and network bandwidth, particularly in
production environments. The streamlined derivations of functional package management,
compared to the larger images produced by containerization technologies, highlight
the potential advantages of functional package management in terms of efficiency.
Furthermore, scalability and deployment speeds are assessed to determine how well
these systems perform when scaled to meet increasing production demands.

Through these objectives, this thesis aims to offer concrete, data-driven recommendations
to organizations and developers, helping them select the most suitable tools for their
development and deployment processes. The goal is to enhance the efficiency, reliability,
and scalability of software workflows while providing a clear comparison of functional
package management and containerization technologies.
