\section{Scope}

This thesis provides a comprehensive analysis of functional package management and
containerization technologies as they are applied throughout the software development
pipeline, focusing on the key stages of development, testing, and deployment. The
research encompasses both development and deployment tools, examining how these
technologies integrate into existing workflows and enhance productivity and reproducibility.
The scope of this study is broad, aiming to evaluate how functional package management
and containerization contribute to software engineering practices, from initial coding
to final production deployments.
In the development phase, the thesis evaluates the role of functional package management
and containerization in supporting developers during the coding process. A key aspect
of this analysis is understanding how each approach manages dependencies and provides
reproducible development environments. Functional package management offers developers
a declarative way to define their environment, ensuring that every team member works
within the same configuration, with immutability and reproducibility at the core of the
approach. On the other hand, containerization technologies encapsulate the entire
development environment within a container, providing process isolation and flexibility
by packaging applications along with their dependencies and operating systems. The thesis
also investigates how well these technologies integrate with popular development tools
and frameworks, such as integrated development environments (IDEs). The ability of these
technologies to seamlessly interface with widely used tools is crucial for their adoption
in real-world projects, where productivity and workflow efficiency are paramount.

Moving into the deployment phase, the thesis explores how functional package management
and containerization are employed to package, deploy, and manage applications in production
environments. This includes assessing their ability to create reproducible deployment
artifacts, which are vital for ensuring consistency between development, staging, and
production environments. The research examines how containerization technologies, using
layered images and orchestration capabilities, enable efficient scaling and management
of applications at scale, particularly with orchestration tools like Kubernetes. Similarly,
functional package management is analyzed for its ability to create reproducible deployment
packages through declarative configuration, ensuring consistency across all stages of the
deployment pipeline. The thesis also addresses the scalability of both technologies,
exploring how well they handle growing demands in production environments, and how they
automate deployment tasks to reduce operational overhead and improve efficiency.

Throughout the analysis, several factors that influence the selection and application of
functional package management and containerization in various software development
contexts are considered. These factors include the specific requirements of the project,
the expertise of the development team, and the constraints imposed by the organization.
For instance, a team with more experience in container orchestration may prefer a
containerization solution, while an organization prioritizing maximum reproducibility and
environmental consistency might favor functional package management. By assessing the
strengths and limitations of both technologies across different stages of the software
development pipeline, the thesis aims to provide insights into their suitability for
various use cases, ranging from small, self-contained projects to large-scale applications
requiring complex orchestration and scalability.

It is important to note that the scope of this thesis does not extend to monitoring and
observability aspects of software systems. While these components are crucial in modern
production environments, they fall outside the focus of this study. Instead, the thesis
maintains its emphasis on the development and deployment phases, focusing on how
these tools impact software engineering practices
in these areas. The goal is to provide a focused examination of how these tools shape
the reproducibility, scalability, and efficiency of software environments without delving
into post-deployment monitoring strategies.

By focusing on how these technologies manage environments,
automate deployments, and scale applications,
the thesis aims to deliver valuable insights for developers, organizations, and researchers
seeking to leverage these technologies in their workflows.
