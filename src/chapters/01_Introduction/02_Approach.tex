\section{Approach}

To systematically address the \hyperlink{research_question}{research question}, this thesis
adopts a comprehensive approach that integrates theoretical analysis, empirical evaluation,
and practical exploration. The methodology begins with an extensive review of existing
literature on functional package management and containerization technologies. This literature
review encompasses academic papers, technical documentation, blog posts, and industry
reports. The purpose of the review is to gain a thorough understanding of the theoretical
foundations and practical uses of both functional package management and containerization
technologies, which are widely used in modern software engineering. The literature review
provides a foundation by examining the history, core principles, and technical developments
behind these paradigms, highlighting how they have evolved to solve specific challenges in
development and deployment. By synthesizing insights from a variety of sources, the literature
review lays the groundwork for further theoretical and empirical analysis.

The theoretical analysis builds upon the literature by exploring the design philosophies
and technical structures of both functional package management and containerization
technologies. In this phase, core elements such as package management, dependency resolution,
environment isolation, and reproducibility are closely examined. Functional package
management emphasizes declarative configurations to build environments, ensuring that each
build is consistent and reproducible. In contrast, containerization technologies encapsulate
entire environments—including applications, dependencies, and the operating system—within
isolated units. Particular attention is given to how these approaches handle immutability,
caching, and dependency conflicts, as these are crucial factors in determining the reliability
and reproducibility of development environments. This theoretical analysis helps establish
a conceptual framework for understanding the unique strengths and limitations of both
functional package management and containerization before moving into empirical testing.

The empirical evaluation is central to this research and involves setting up a series of
controlled experiments to measure the performance, scalability, and efficiency of functional
package management and containerization technologies in real-world development and deployment
scenarios. These experiments are designed to evaluate how each technology performs under
typical conditions faced by development teams. A key focus is placed on measuring build
times, deployment speeds, and the size of deployment artifacts. For instance, build times
are assessed under cold builds, warm builds, and fully cached builds to reflect various
stages of a developer’s workflow. Both technologies are tested using a web server project
written in Rust, and the experiments are automated through GitHub Actions, ensuring consistent
conditions for each test run. The layered caching model of containerization is compared
to the derivation-based system of functional package management, with metrics such as time
saved from caching, overall build speed, and the impact of caching on warm and cold builds
closely monitored. Additionally, the final package sizes produced by both approaches are
analyzed to determine which technology generates more efficient artifacts for deployment
in terms of storage and bandwidth. By carefully controlling the experimental conditions,
the empirical evaluation generates reliable, reproducible data that supports the theoretical
analysis conducted earlier.

Finally, the findings from both the theoretical analysis and empirical evaluation are
synthesized in a discussion that critically addresses the \hyperlink{research_question}
{research question}. This discussion not only compares the performance of functional
package management and containerization technologies but also explores the practical
implications for development workflows, deployment processes, and overall project
management. The strengths, weaknesses, opportunities, and risks of each approach are
analyzed, offering valuable insights into when it is most appropriate to adopt one
technology over the other. The discussion integrates findings to provide practical
recommendations for developers, teams, and organizations aiming to optimize their software
development and deployment workflows.

This multifaceted approach aims to provide a thorough and nuanced exploration of the
research question. By combining a detailed theoretical framework with rigorous empirical
testing, the thesis offers a comprehensive comparison of functional package management
and containerization technologies, highlighting their similarities, differences, and
broader implications for modern software engineering practices.
