\chapter{Introduction and Motivation}

In the rapidly evolving field of software development,
managing development environments and ensuring consistent deployment
has become increasingly challenging \cite{ContainerTechnology}.
This thesis explores functional package management and containerization,
two key concepts that address these challenges.

Functional package management creates reproducible
and consistent software environments through a declarative approach.
It ensures that software components
and their dependencies are precisely defined, reducing conflicts and errors.
This method is crucial for projects where reproducing the exact development environment is essential
for debugging, testing, and deployment \cite{malkaIncreasingTrustOpen2024}.

Containerization, on the other hand, packages applications
and their dependencies into portable containers.
This ensures that applications run consistently across various environments,
from development to production.
Containers provide isolated environments for applications,
simplifying deployment and enhancing scalability \cite{pahlContainerizationPaaSCloud2015}.

The relevance of this thesis lies in examining these foundational concepts,
which are vital in modern software development.
As organizations adopt
DevOps \cite{DevOps2024} practices
and CI/CD \cite{CICD2024} pipelines,
the need for reliable tools to manage environments and deployments grows \cite{ContainerTechnology}.
Understanding functional package management
and containerization helps developers and organizations make informed decisions about their tools and methodologies.

The motivation for this thesis comes from the increasing complexity of software projects
and the challenges of managing development environments.
Issues like "dependency hell" \cite{DependencyHell2024}
and environment discrepancies can slow development, introduce bugs, and complicate deployment.
This thesis aims to provide solutions to these problems
by exploring functional package management and containerization.

Reproducibility is a significant driver for this study.
In complex projects, reliably recreating the development environment is crucial.
Functional package management ensures reproducibility,
while containerization guarantees consistent and portable deployment,
maintaining stability across different stages of the software lifecycle \cite{malkaIncreasingTrustOpen2024}.

This thesis offers a comprehensive understanding of functional package management
and containerization, filling a gap in existing literature.
The insights gained will help developers, system administrators,
and organizations choose the best tools for their needs.
As the software development landscape evolves,
understanding these concepts will be essential for managing environments and deployments effectively.

In conclusion, this thesis addresses the challenges of modern software development
by exploring functional package management and containerization.
By enhancing the efficiency, reliability, and scalability of development and deployment processes,
this work contributes valuable insights to both academic and industry audiences.

\section{Objectives}

The primary goal of this thesis is to provide a comprehensive comparison between
functional package management and containerization technologies as development and
deployment tools. The thesis seeks to analyze these two paradigms through empirical
and practical exploration, focusing on key metrics such as build times, deployment
speeds, package sizes, and the management of developer environments.

At the heart of this research is the central question: \textbf{How do functional package
	management and containerization technologies compare as development and deployment
	tools, and what recommendations can be derived for enterprises that use these
	technologies?} The purpose of this research is to offer insights that help developers
and organizations make informed decisions about which technology to adopt depending
on their specific needs.

In order to thoroughly explore this question, the thesis pursues several objectives.
First, it aims to examine the philosophical underpinnings of both functional package
management and containerization. Functional package management emphasizes immutability,
reproducibility, and declarative configurations. Containerization technologies, on
the other hand, center on process isolation, portability, and flexibility. These core
principles are vital to understanding the broader context in which each paradigm
excels. It is essential to investigate how these philosophies impact the setup and
management of development environments, software reproducibility, and the overall
software lifecycle.

The second objective is to conduct a technical analysis of the implementation details
that differentiate these technologies. By evaluating their declarative configurations
and lifecycle management, the thesis explores how these approaches provide strengths
and expose limitations, particularly in areas like environment isolation, package
dependencies, and configuration complexity. These technical distinctions reveal the
potential benefits and constraints of using either functional package management or
containerization technologies in different development contexts.

The third objective is to apply these technologies to a real-world project to provide
empirical evidence of their effectiveness. A web server written in Rust serves as the
test bed for demonstrating how functional package management and containerization
behave under practical conditions, with a focus on dependency management, environment
isolation, build reproducibility, and deployment. This case study illustrates how each
technology can handle typical tasks faced by development teams and highlights specific
scenarios where one approach may be more suitable than the other.

Finally, the thesis evaluates performance metrics such as build times, package sizes,
and deployment scalability. Build times are crucial because they directly affect
developer productivity and the continuous integration/continuous deployment (CI/CD)
pipeline. By comparing containerization's layered caching approach with the derivation-based
build system of functional package management, the research examines how quickly each
system can produce artifacts under various conditions. Package size, another key metric,
has significant implications for storage and network bandwidth, particularly in
production environments. The streamlined derivations of functional package management,
compared to the larger images produced by containerization technologies, highlight
the potential advantages of functional package management in terms of efficiency.
Furthermore, scalability and deployment speeds are assessed to determine how well
these systems perform when scaled to meet increasing production demands.

Through these objectives, this thesis aims to offer concrete, data-driven recommendations
to organizations and developers, helping them select the most suitable tools for their
development and deployment processes. The goal is to enhance the efficiency, reliability,
and scalability of software workflows while providing a clear comparison of functional
package management and containerization technologies.

\section{Approach}

To systematically address the \hyperlink{research_question}{research question}, this thesis
adopts a comprehensive approach that integrates theoretical analysis, empirical evaluation,
and practical exploration. The methodology begins with an extensive review of existing
literature on functional package management and containerization technologies. This literature
review encompasses academic papers, technical documentation, blog posts, and industry
reports. The purpose of the review is to gain a thorough understanding of the theoretical
foundations and practical uses of both functional package management and containerization
technologies, which are widely used in modern software engineering. The literature review
provides a foundation by examining the history, core principles, and technical developments
behind these paradigms, highlighting how they have evolved to solve specific challenges in
development and deployment. By synthesizing insights from a variety of sources, the literature
review lays the groundwork for further theoretical and empirical analysis.

The theoretical analysis builds upon the literature by exploring the design philosophies
and technical structures of both functional package management and containerization
technologies. In this phase, core elements such as package management, dependency resolution,
environment isolation, and reproducibility are closely examined. Functional package
management emphasizes declarative configurations to build environments, ensuring that each
build is consistent and reproducible. In contrast, containerization technologies encapsulate
entire environments—including applications, dependencies, and the operating system—within
isolated units. Particular attention is given to how these approaches handle immutability,
caching, and dependency conflicts, as these are crucial factors in determining the reliability
and reproducibility of development environments. This theoretical analysis helps establish
a conceptual framework for understanding the unique strengths and limitations of both
functional package management and containerization before moving into empirical testing.

The empirical evaluation is central to this research and involves setting up a series of
controlled experiments to measure the performance, scalability, and efficiency of functional
package management and containerization technologies in real-world development and deployment
scenarios. These experiments are designed to evaluate how each technology performs under
typical conditions faced by development teams. A key focus is placed on measuring build
times, deployment speeds, and the size of deployment artifacts. For instance, build times
are assessed under cold builds, warm builds, and fully cached builds to reflect various
stages of a developer’s workflow. Both technologies are tested using a web server project
written in Rust, and the experiments are automated through GitHub Actions, ensuring consistent
conditions for each test run. The layered caching model of containerization is compared
to the derivation-based system of functional package management, with metrics such as time
saved from caching, overall build speed, and the impact of caching on warm and cold builds
closely monitored. Additionally, the final package sizes produced by both approaches are
analyzed to determine which technology generates more efficient artifacts for deployment
in terms of storage and bandwidth. By carefully controlling the experimental conditions,
the empirical evaluation generates reliable, reproducible data that supports the theoretical
analysis conducted earlier.

Finally, the findings from both the theoretical analysis and empirical evaluation are
synthesized in a discussion that critically addresses the \hyperlink{research_question}
{research question}. This discussion not only compares the performance of functional
package management and containerization technologies but also explores the practical
implications for development workflows, deployment processes, and overall project
management. The strengths, weaknesses, opportunities, and risks of each approach are
analyzed, offering valuable insights into when it is most appropriate to adopt one
technology over the other. The discussion integrates findings to provide practical
recommendations for developers, teams, and organizations aiming to optimize their software
development and deployment workflows.

This multifaceted approach aims to provide a thorough and nuanced exploration of the
research question. By combining a detailed theoretical framework with rigorous empirical
testing, the thesis offers a comprehensive comparison of functional package management
and containerization technologies, highlighting their similarities, differences, and
broader implications for modern software engineering practices.

\section{Scope}

This thesis provides a comprehensive analysis of functional package management and
containerization technologies as they are applied throughout the software development
pipeline, focusing on the key stages of development, testing, and deployment. The
research encompasses both development and deployment tools, examining how these
technologies integrate into existing workflows and enhance productivity and reproducibility.
The scope of this study is broad, aiming to evaluate how functional package management
and containerization contribute to software engineering practices, from initial coding
to final production deployments.
In the development phase, the thesis evaluates the role of functional package management
and containerization in supporting developers during the coding process. A key aspect
of this analysis is understanding how each approach manages dependencies and provides
reproducible development environments. Functional package management offers developers
a declarative way to define their environment, ensuring that every team member works
within the same configuration, with immutability and reproducibility at the core of the
approach. On the other hand, containerization technologies encapsulate the entire
development environment within a container, providing process isolation and flexibility
by packaging applications along with their dependencies and operating systems. The thesis
also investigates how well these technologies integrate with popular development tools
and frameworks, such as integrated development environments (IDEs). The ability of these
technologies to seamlessly interface with widely used tools is crucial for their adoption
in real-world projects, where productivity and workflow efficiency are paramount.

Moving into the deployment phase, the thesis explores how functional package management
and containerization are employed to package, deploy, and manage applications in production
environments. This includes assessing their ability to create reproducible deployment
artifacts, which are vital for ensuring consistency between development, staging, and
production environments. The research examines how containerization technologies, using
layered images and orchestration capabilities, enable efficient scaling and management
of applications at scale, particularly with orchestration tools like Kubernetes. Similarly,
functional package management is analyzed for its ability to create reproducible deployment
packages through declarative configuration, ensuring consistency across all stages of the
deployment pipeline. The thesis also addresses the scalability of both technologies,
exploring how well they handle growing demands in production environments, and how they
automate deployment tasks to reduce operational overhead and improve efficiency.

Throughout the analysis, several factors that influence the selection and application of
functional package management and containerization in various software development
contexts are considered. These factors include the specific requirements of the project,
the expertise of the development team, and the constraints imposed by the organization.
For instance, a team with more experience in container orchestration may prefer a
containerization solution, while an organization prioritizing maximum reproducibility and
environmental consistency might favor functional package management. By assessing the
strengths and limitations of both technologies across different stages of the software
development pipeline, the thesis aims to provide insights into their suitability for
various use cases, ranging from small, self-contained projects to large-scale applications
requiring complex orchestration and scalability.

It is important to note that the scope of this thesis does not extend to monitoring and
observability aspects of software systems. While these components are crucial in modern
production environments, they fall outside the focus of this study. Instead, the thesis
maintains its emphasis on the development and deployment phases, focusing on how
these tools impact software engineering practices
in these areas. The goal is to provide a focused examination of how these tools shape
the reproducibility, scalability, and efficiency of software environments without delving
into post-deployment monitoring strategies.

By focusing on how these technologies manage environments,
automate deployments, and scale applications,
the thesis aims to deliver valuable insights for developers, organizations, and researchers
seeking to leverage these technologies in their workflows.

